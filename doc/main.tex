\documentclass[a4paper]{article}
\author{Hiroaki Yamamoto}
\title{Creating a simulator for KUE-chip2}

\begin{document}
    \maketitle
    \section{Requirements}
        \begin{itemize}
            \item Basically, this simulator works as an interpretor for a specified machine language file.
            \item The simulator needs machine language file to input program area.
            \item A user can input data into data area by choosing a file.\\
                  In addtion to this, the user can input initial state for registers/flags.
            \item To avoid infinite loop, the user can specify step-limitation.
            \item The result must be stored into the specified file name as comma separated values.
            \item Those above options are provided via command line options. The details are section \ref{Options}.
        \end{itemize}
    \section{Options\label{Options}}
        \begin{table}[htb]
            \begin{center}
                \caption{Command line options}
                \begin{tabular}{|l|l|l|c|l|p{3cm}|}
                    \hline Option&Alias&Argument&Attribute&Default value&Description\\
                    \hline \verb@--initial-state@&-ii&File Name&Optional&&Sets an initial registers/flags state\\
                    \hline \verb@--data-area@&-d&File Name&Optional&&Sets data into data area.\\
                    \hline \verb@--step@&-s&Number&Optional&&Sets step-limitation\\
                    \hline \verb@--output@&-o&Number&Optional&result.csv&Sets an output file.\\
                    \hline \verb@--input@&-i&File Name&Needed&&Sets a machine language file.\\
                    \hline \verb@--per-inst@&-p&none&Optional&&Interactive Mode.\\
                    \hline
                \end{tabular}
            \end{center}
        \end{table}
        Note that \verb@--input@ option is Default Option. i.e. If you execute the simulator with the following, the command line argument is interpreted as \verb@--input@:
        \begin{verbatim}
./simulator program.bin
        \end{verbatim}
        \clearpage
    \section{Data formats}
        \subsection{Program data (Machine language file)}
        The program data is stored as it is. Note that the data is {\bf BINARY data hand-assembled}.
        
        \subsection{Data area file}
        The data area file is stored as it is. Uninitialized data is interperted as 0x1f. The file size is smaller than 256 bytes, each extra area is set to 0x1f.
        
        \subsection{Initial Sates for Registers/flags}
        The data format is the following:
        
        \begin{table}[htb]
                \begin{center}
                    \caption{Data structure of Initial States for Refisters/flags \label{structure}}
                    \begin{tabular}{|c|c|c|c|c|c|c|c|}
                        \hline0-7&8-15&16-24&25&26&27&28&29-32\\
                        \hline ACC&IX&OBUF&CF&VF&NF&ZF&0\\
                        \hline
                    \end{tabular}
                \end{center}
        \end{table}
        
        The unit of Table.\ref{structure} is bit.
        
        \subsection{Output file}
        Output file is stored into a file as Comma Separeted Values with the following schema:
        \begin{verbatim}
        CPUID;Object code;Mnemonic code;PC;ACC;IX;CF;VF;NF;ZF;IN;OUT;MEMORY
        \end{verbatim}

    \section{Data Structure}
        The data structure passing each function is the following.
        \begin{verbatim}
        enum changed_flag{NO_MODIFIED,PROGRAM_AREA,DATA_AREA};
        struct buf{
            unsigned char bits;
            int flag;
        } typedef buf;
        
        struct data{
            unsigned short cpuid;
            unsigned char obj_code[2];
            size_t code_size;
            char *mnemonic_code;
            unsigned char pc,acc,ix;
            unsigned char flags;
            buf *in,out;
            unsigned char *program_memory;
            unsigned char *data_memory;
            //the value is tri-state. look at changed_flag.
            int memory_changed;
            unsigned char modified_addr;
            unsigned char prev;
            unsigned char now;
        } typedef data;
        \end{verbatim}

        Note that \verb@mnemonic_code@ must be allocated dinamically. Moreover, the data will be passed as a pointer to reduce memory usage. "flags" variable has values below 4 bits, and the structure is the following:
        
        \begin{table}[htb]
            \begin{tabular}{|c|c|c|c|}
            \hline 3&&&0\\
            \hline CF&VF&NF&ZF\\
            \hline
            \end{tabular}
        \end{table}

    \section{Output}
        The format to output to stdout is the following:
        \begin{verbatim}
{Object Code} {Mnemonic code} [CPU:{CPU} PC:{PC} ACC:{ACC} IX{IX}\
 CF={CF} VF={VF} ZF={ZF} NF={NF} IN={IN} OUT={OUT} MEMORY={memory diff}]
        \end{verbatim}

        {string} is a variable corresponding to a register/flags.

        Note that memory diff mus be a diff of memory. For example, if the address 0x10 before modified was 0xff and changed to 0x00, the {memory\_diff} syntax is the following:

        \begin{verbatim}
{address}:{the value before modified}->{the value after modified}
        \end{verbatim}
        
        The meanings of other variables are the following:
        
        \begin{table}[htb]
            \begin{center}
                \caption{The meaning of each variable}
                \begin{tabular}{|l|c|}
                    \hline \{PC\}&Program Counter\\
                    \hline \{ACC\}&Accumulator\\
                    \hline \{IX\}&Index Register\\
                    \hline \{CF\}&Carry Flag\\
                    \hline \{VF\}&Overflow Flag\\
                    \hline \{ZF\}&Zero Flag\\
                    \hline \{NF\}&Negative Flag\\
                    \hline \{IN\}&IBUF\\
                    \hline \{OUT\}&OBUF\\
                    \hline \{CPU\}&CPUID\\
                    \hline
                \end{tabular}
            \end{center}
        \end{table}
        
        Note that those above variables are hexadecimal, not string.
\end{document}